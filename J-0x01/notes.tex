\documentclass{article}

\usepackage[english]{babel}
\usepackage[utf8]{inputenc}
\usepackage{times}


\begin{document}

\centerline{\Large J-0x01 - Introduction}
\vspace{5mm}

\textbf{child block image}

Why ML?
\begin{itemize}
\item Playing with blocks vs doing maths
\item Sometimes we know how to solve problems (e.g., sorting)
\item Sometimes we don't (e.g., recognise a cat, read handwriting on envelopes)
\item Not magic
\end{itemize}

What is ML?
\begin{enumerate}
\item Some algorithms we know how to write
  \begin{enumerate}
  \item Sort numbers
  \item Fly a plane
  \end{enumerate}
\item Some algorithms we don't know how to write (example: drive a car)
  \begin{enumerate}
  \item Drive a car 
  \item Read addresses on envelopes
  \item Detect spam
  \end{enumerate}
\item Maybe we can write programs to write programs when we can't
\item Some terms we used to use for ML
  \begin{itemize}
  \item Artificial intelligence
  \item Expert systems
  \end{itemize}
\end{enumerate}

\textbf{fist image}

Disclaimers
\begin{itemize}
\item The literature is overwhelmingly in English
\item Time is short
\end{itemize}

\textbf{tree image}

Types of ML
\begin{enumerate}
\item \fbox{Supervised}
  \begin{enumerate}
  \item Training data: input and correct responses
  \item Regression (continuous) (example: home prices)
  \item Classification (discrete) (example: medical outcome (alive/dead))
  \end{enumerate}
\item \fbox{Unsupervised}
  \begin{enumerate}
  \item Clustering
  \item Deep neural networks
  \item Associative (example: human experience, e.g. from a career)
  \item Dimensionality reduction
  \end{enumerate}
\item \fbox{Reinforcement}
  \begin{enumerate}
  \item Make a choice, get feedback
  \item Online
  \item Can be stochastic (example: predicting weather from local clues)
  \end{enumerate}
\end{enumerate}

\textbf{high five image}

Talk about course structure
\begin{itemize}
\item Ten afternoon sessions
\item In class: mostly theory, some code, not much maths
\item Between classes: coding assignments (python)
\item Communication: github, mailing list
\item Ask questions via mail, eventually github issues
\item Help each other via same
\item Don't copy.  Learn.
\end{itemize}

\fbox{Curse of Dimensionality}
\begin{enumerate}
\item \textit{Fléau (ou : malédiction) de la dimension}
\item Volume of unit cube $\pm\epsilon$
\item Distance from $(0,0,\ldots,0)$ to $(1,1,\ldots,1)$
\item Physics: $1/r^{d-1}$
\item It's easy to get lost\dots
\item Richard Ernest Bellman, Dynamic programming, Princeton
  University Press, 1957.
\end{enumerate}

Probability
\begin{enumerate}
\item Event
\item Complement of an event
\item Disjoint (mutually exclusive)
\item Independent events --- knowing one outcome gives no information about other
\item \fbox{addition ($\times 2$), multiplication $\times 2$)}
\item \fbox{Conditional probability}
\item Marginal probability
\item Joint probability
\end{enumerate}

Statistics
\begin{enumerate}
\item Goal for a bit: think like a statistician
\item \fbox{What is statistics? ($\times 3$)}
\item Said differently: goal is to compare reality to a model
\item Or to find a model and then compare.
\item Good statistical models are often relatively simple.
\item \fbox{What is data science ($\times 5$)}
\end{enumerate}

Study design
\begin{enumerate}
\item \fbox{Anecdote}
\item \fbox{Study types ($\times 2$)}
\item Observational studies can't conclude causality
\item Observational studies can be
  \begin{itemize}
  \item prospective: identify individuals, collect information
  \item retrospective
  \item we can combine them
  \end{itemize}
\item Experimental studies
  \begin{itemize}
  \item We do stuff
  \item Can conclude causation if properly designed
    \begin{itemize}
    \item controlling: hold other variables constant (e.g., drink pill
      with full glass of water even if we don't care)
    \item randomization: cancel out effects we can't control
    \item replication: enough participants
    \end{itemize}
  \end{itemize}
\item Study types example
  \begin{itemize}
  \item Sunscreen use correlated to skin cancer rates.
  \item Confounding variable
  \end{itemize}
\item Random sampling hazards
  \begin{itemize}
  \item Not actually random
  \item Convenience sample
  \item Non-response bias
  \end{itemize}
\end{enumerate}

Statistical concepts
\begin{enumerate}
\item \fbox{Variable types}
  \begin{itemize}
  \item Input:  Features
  \item Input variables measure:  Explanatory variable
  \item Output: Response variable
  \item Training set
  \item Test set (tune parameters) (compare model parameters)
  \item Validation set (tune hyperparameters) (measure performance of model)
  \item Cross validation
  \item Bias - same errors regardless of input (inflexible)
  \item Variance - different errors with same input (too flexible)
  \end{itemize}
\item \fbox{Population statistics ($\times 6$)}
  \begin{itemize}
  \item sample mean vs population \fbox{mean ($\times 7$)}
  \item Sample standard deviation and variance: divide by $n-1$
  \end{itemize}
\item Distributions
  \begin{itemize}
  \item \fbox{Important: pdf (pmf), cdf, ppf ($\times 5$)}
    \begin{itemize}
    \item pdf = densité de probabilité
    \item pmf = fonction de masse
    \item cdf = fonction de répartition
    \item ppf = ?
    \end{itemize}
  \item The rest: just so you've heard of them
  \item \fbox{Boxplot ($\times 2$)}
  \end{itemize}
\item \fbox{Normal distributions ($\times 2$)}
  \begin{itemize}
  \item Sample mean vs population mean
  \item How close are they?
  \item Point estimate: if you have to guess, this is it
  \item Correction: if I want to be on average weighted right as much possible
  \end{itemize}
\item \fbox{Sampling distributions ($\times 3$)}
  \begin{itemize}
  \item Sampling mean is unimodal and approximately symmetric
  \item It is centred at population mean.
    \item The standard deviation of the sample mean tells us how far a
      point sample's mean is likely to be from the population mean.
      In other words, how much error we are likely to have in the
      point estimate's mean.  \textbf{Standard error.}
    \item TODO: Generate uniform population, sample,
      and plot sampling distribution
    \item TODO Generate highly skewed population, sample, and plot
      sampling distribution
    \item In real life, we don't have access to the population
      parameters.  We have to \textit{estimate} them from samples.  So
      we can't \textit{know} the standard error (erreur type).
  \end{itemize}
\item \fbox{Confidence intervals ($\times 3$)}
  \begin{itemize}
  \item Sampling is usually expensive.
  \item Reminder:  Independent random samples!
  \item
    Correct language: ``We are 95\% confident that the population
    parameter is between\dots''
  \item
    Incorrect language: describe the confidence interval as capturing
    the population parameter with a certain probability.
  \item
    This is one of the most common errors: while it might be useful to
    think of it as a probability, the confidence level only quantifies
    how plausible it is that the parameter is in the interval.
  \item
    Another especially important consideration of confidence intervals
    is that they only try to capture the population parameter. Our
    intervals say \textit{nothing} about the confidence of
    \begin{itemize}
    \item capturing individual observations
    \item a proportion of the observations
    \item about capturing point estimates
    \end{itemize}
    Confidence intervals only attempt to capture population
    parameters.
  \end{itemize}
\end{enumerate}

\vspace{10mm}
\centerline{\Large J-0x01 - Features and Modeling}
\vspace{5mm}


\end{document}
