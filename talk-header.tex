\documentclass[t,aspectratio=169]{beamer}

\usepackage[english]{babel}
\usepackage[utf8]{inputenc}
\usepackage{times}
\usepackage[T1]{fontenc}
\usepackage{url}
\usepackage{relsize}
\usepackage{amsmath,mathtools}  % Mathtools provides dcase.
%\usepackage{eulervm}            % Euler VM fonts (for math mode) (but makes \hat not work)
%\usepackage[scaled]{helvet}
\usepackage{listings}
\usepackage{mathabx}            % Provides nicer \land.
\usepackage{xcolor}
\usepackage{graphicx}
\usepackage{physics}

\lstset{ %
  basicstyle=\small\ttfamily,
  moredelim=[is][\textcolor{blue}]{|}{|}
}

\newcommand\topstrut{\rule{0pt}{2.6ex}}
\newcommand\bottomstrut{\rule[-1.2ex]{0pt}{0pt}}
\newcommand\doublestrut{\rule[-1.2ex]{0pt}{3.6ex}}

\newcommand\blue[1]{\textcolor{blue}{#1}}
\newcommand\red[1]{\textcolor{red}{#1}}
\newcommand\green[1]{\textcolor{green}{#1}}
\newcommand\gray[1]{\textcolor{gray}{#1}}
\newcommand\purple[1]{\textcolor{purple}{#1}}
\newcommand\smallgray[1]{\textcolor{gray}{\footnotesize\it #1}}
\newcommand\prevwork[1]{\smallgray{#1}}
\newcommand\solo[1]{\vfill\centerline{#1}}
\newcommand\soloo[2]{\only<#1>{\solo{#2}}}
\newcommand\solopb[1]{\vfill\centerline{\parbox{.9\textwidth}{#1}}}
\newcommand\soloopb[2]{\only<#1>{\solopb{#2}}}
\newcommand\cimgo[1]{\vfill\centerline{\includegraphics{#1}}\vfill}
\newcommand\cimgw[1]{\vfill\centerline{\includegraphics[width=\paperwidth]{#1}}\vfill}
\newcommand\cimgwb[1]{\centerline{\includegraphics[width=\paperwidth]{#1}}}
\newcommand\cimg[1]{\vfill\centerline{\includegraphics[width=.9\textwidth]{#1}}\vfill}
\newcommand\cimgg[1]{\vfill\centerline{\includegraphics[width=.8\textwidth]{#1}}\vfill}
\newcommand\cimggg[1]{\vfill\centerline{\includegraphics[width=.7\textwidth]{#1}}\vfill}
\newcommand\cimgsm[1]{\vfill\centerline{\includegraphics[width=.4\textwidth]{#1}}\vfill}
\newcommand\cimgt[1]{\centerline{\includegraphics[width=.2\textwidth]{#1}}\vfill}
\newcommand\cimgh[1]{\vfill\centerline{\includegraphics[height=.9\textheight]{#1}}\vfill}
\newcommand\cimghh[1]{\vfill\centerline{\includegraphics[height=.8\textheight]{#1}}\vfill}
\newcommand\cimghhh[1]{\vfill\centerline{\includegraphics[height=.7\textheight]{#1}}\vfill}
\newcommand\cimghhhh[1]{\vfill\centerline{\includegraphics[height=.6\textheight]{#1}}\vfill}

% The differential in an integral.
% After a function or a fraction, the \, may not be desired, see \DD.
% It is as much art, taste, and consistency as norms and science.
\newcommand\corr[0]{\mathbf{Corr}}
\newcommand\cov[0]{\mathbf{Cov}}
\newcommand\D[1]{\,\mathrm{d}{#1}}
\newcommand\DD[1]{\mathrm{d}{#1}}
\newcommand\E[0]{\mathbf{E}}
%\newcommand\var[0]{\mathbf{Var}}
\newcommand\N[0]{\mathcal{N}}
\newcommand\R[0]{\mathbb{R}}

% Only useful in beamer.
\newcommand\talksection[1]{\section{#1}
\begin{frame}
  \vfill
  \Huge\bf\blue{\centerline{#1}}
  \vfill
\end{frame}
}

\newcommand\startsession[1]{\centerline{\Large #1}}
\newcommand\phrase[1]{\vspace{1cm}\centerline{\large\bf\blue{#1}}}
\newcommand\vphrase[1]{\vfill\centerline{\large\bf\blue{#1}}\vfill}
\newcommand\sphrase[1]{\vspace{1cm}\centerline{\large\blue{#1}}}
\newenvironment{bphrase}
  {\blue\bgroup}
  {\egroup}
\newenvironment{mphrase}
  {\begin{displaymath}\blue\bgroup}
  {\egroup\end{displaymath}}
\newcommand\wphrase[1]{\vspace{1cm}\centerline{\large\bf\textcolor{white}{#1}}}


\mode<presentation>
{
  \usetheme{default}
  \setbeamertemplate{navigation symbols}{}
  \setbeamertemplate{footline}[frame number]
  \setbeamertemplate{items}[circle]
  \usecolortheme{seahorse}
}

\author[Abrahamson] {Jeff Abrahamson}
\date{Cours sur l'année, 2017--2018}
%\institute{}

%\setbeamertemplate{footline}[text line]{%
%  \parbox{\linewidth}{\vspace*{-8pt}some text\hfill\insertshortauthor\hfill\insertpagenumber}}
\setbeamertemplate{footline}[text line]{%
  \parbox{0.8\linewidth}{
    %\vspace*{-8pt}\insertshorttitle~(\insertshortauthor)
    \vspace*{-8pt}Copyright 2017--2018 Jeff Abrahamson, for private use by course students only
  }
  \hfill%
  \parbox{.2\linewidth}{\vspace*{-8pt}{ML Week}}
  \hfill%
  \parbox{0.15\linewidth}{
    \vspace*{-8pt}\raggedleft\insertpagenumber
  }
}

\newcommand\cnote[1]{}

\parskip=8 pt
