\input ../talk-header.tex
\title{Machine Learning}
\subtitle{Faces, Clustering, Anomalies}

% If you wish to uncover everything in a step-wise fashion, uncomment
% the following command: 
%\beamerdefaultoverlayspecification{<+->}

\begin{document}

\begin{frame}
  \titlepage
\end{frame}

%%%%%%%%%%%%%%%%%%%%%%%%%%%%%%%%%%%%%%%%%%%%%%%%%%%%%%%%%%%%%%%%%%%%%%
%%%%%%%%%%%%%%%%%%%%%%%%%%%%%%%%%%%%%%%%%%%%%%%%%%%%%%%%%%%%%%%%%%%%%%
%%%%%%%%%%%%%%%%%%%%%%%%%%%%%%%%%%%%%%%%%%%%%%%%%%%%%%%%%%%%%%%%%%%%%%

\talksection{Review}

\begin{frame}
  \phrase{Linear Regression}
\end{frame}

\begin{frame}
  \phrase{Logistic Regression}
\end{frame}

\begin{frame}
  \phrase{SVM}
\end{frame}

\begin{frame}
  \phrase{CART}
\end{frame}

\begin{frame}
  \phrase{Random Forests}
\end{frame}

\begin{frame}
  \phrase{PCA}
\end{frame}

\talksection{PCA}

\begin{frame}
  \phrase{Principle component analysis}

  \sphrase{Analyse en composantes principales}
\end{frame}

\begin{frame}
  \frametitle{Motivation}
  \phrase{Remember the Curse of Dimensionality?}
\end{frame}

\begin{frame}[t]
  \frametitle{Principle}

  \vspace{1cm}
  \begin{itemize}
  \item Linear transformations have axes
  \item Find them (eigenvectors of the covariance matrix)
  \item Pick the biggest ones
  \end{itemize}

  \vspace{5mm}
  \only<2>{\phrase{Fitting an $n$-dimensional ellipsoid to the data}}
\end{frame}

\begin{frame}
  \frametitle{Uses}
  \begin{itemize}
  \item Exploratory data analysis
  \item Compression
  \end{itemize}
\end{frame}

\begin{frame}
  \frametitle{Also known as}
  \begin{itemize}
  \item Discrete Kosambi-Karhunen–Loève transform (KLT) (signal processing)
  \item Hotelling transform (multivariate quality control)
  \item Proper orthogonal decomposition (POD) (ME)
  \item Singular value decomposition (SVD), Eigenvalue decomposition (EVD) (linear algebra)
  \item Etc.
  \end{itemize}
\end{frame}

\begin{frame}
  \frametitle{History}
  \begin{itemize}
  \item Invented by Karl Pearson in 1901
  \item Invented (again) and named by Harold Hotelling in 1930's
  \item Also known as\dots
  \end{itemize}
\end{frame}

\begin{frame}
  \frametitle{Also known as}

  \begin{itemize}
  \item It's a long list, every field uses a different name\dots
  \end{itemize}
\end{frame}


\talksection{Face Recognition}

\begin{frame}
  \frametitle{Eigenfaces}
  \only<1>{
    \begin{itemize}
    \item Sirovich and Kirby (1987)
    \item Turk and Pentland (1991)
    \end{itemize}
    \vfill
    \prevwork{Turk, Matthew A and Pentland, Alex P. Face recognition
      using eigenfaces. Computer Vision and Pattern Recognition,
      1991. Proceedings {CVPR'91.}, {IEEE} Computer Society Conference
      on 1991.}
  }
  \only<2>{ Want: a low-dimensional representation of a face

    Plan: cluster simplified faces
  }
  \only<3>{
    Viewed as compression:
    \begin{itemize}
    \item Use PCA on face images to form a set of basis features
    \item Use eigenpictures to reconstruct original faces
    \end{itemize}
  }
  \only<4>{
    \cimghh{eigenface-reconstruction-opencv.png}
  }
\end{frame}

\begin{frame}
  \frametitle{Eigenfaces algorithm}
  
  \only<1>{
    Let $X=\{x_1, x_2, \dotsc, x_n\}$ be a random vector with observations $x_i\in\mathbb{R}^d$.

    Compute
    
    \begin{displaymath}
      \mu=\frac{1}{n} \sum_{i=1}^n x_i
    \end{displaymath}

    \vfill
    \prevwork{OpenCV}
  }
  \only<2>{
    Compute the covariance matrix $S$:
    
    \begin{displaymath}
      S = \frac{1}{n} \sum_{i=1}^n (x_i-\mu)(x_i-\mu)^T
    \end{displaymath}
  }
  \only<3-4>{
    Compute the eigenvectors of $S$:
    
    \begin{displaymath}
      Sv_i = \lambda_i v_i \qquad i=1,2,\dotsc, n
    \end{displaymath}

    Sort the eigenvectors in decreasing order.

    We want the $k$ principal components, so take the first $k$.

    \only<4>{\blue{This is PCA.}}
  }
  \only<5>{
    The $k$ principal components of the observed vector $x$ are then given by
    
    \begin{displaymath}
      y=W^T(x-\mu)
    \end{displaymath}
    
    where
    \begin{displaymath}
      W = \begin{bmatrix}
        \vline & \vline & & \vline \\
        v_1& v_2 & \cdots & v_k \\
        \vline & \vline & & \vline
      \end{bmatrix}
    \end{displaymath}
  }
  \only<6>{
    The reconstruction from the PCA basis is then
    
    \begin{displaymath}
      x = Wy + \mu
    \end{displaymath}
  }
  \only<7>{
    So the plan is this:
    \begin{itemize}
    \item Project all training samples in the PCA subspace
    \item Project the query into the PCA subspace
    \item Find the nearest neighbour to the projected query image among
      the projected training images
    \end{itemize}
  }
  \only<8>{
    \cimghh{eigenface-reconstruction-opencv.png}
  }
  \only<9>{
    Some advantages:
    \begin{itemize}
    \item Easy, relatively inexpensive
    \item Recognition cheaper than preprocessing
    \item Reasonably large database possible
    \end{itemize}
  }
  \only<10>{
    Some problems:
    \begin{itemize}
    \item Need controlled environment
    \item Needs straight-on view
    \item Sensitive to expression changes
    \item If lots of variance is external (e.g., lighting)\dots
    \end{itemize}
  }
\end{frame}


\talksection{Handwriting Recognition}

\begin{frame}
  \frametitle{Introduction to Handwriting Recognition}
  \only<1>{
    Choices
    \begin{itemize}
    \item Online
    \item Offlne
    \end{itemize}
  }
  \only<2>{
    Choices
    \begin{itemize}
    \item Get path information
    \item Get time data
    \item Get pressure information
    \item Only get image
    \end{itemize}
  }
  \only<3>{
    Major techniques
    \begin{itemize}
    \item Clustering (not great performance)
    \item SVM (until 2006 or so)
    \item Convolutional neural networks
    \end{itemize}
  }
\end{frame}

%%%%%%%%%%%%%%%%%%%%%%%%%%%%%%%%%%%%%%%%%%%%%%%%%%%%%%%%%%%%%%%%%%%%%%
%\talksection{Break}

\begin{frame}
  % https://www.pexels.com/photo/black-african-ethnicity-person-black-and-white-27752/
  % https://static.pexels.com/photos/27752/pexels-photo.jpg
  % CC0 license
  \cimgwb{man.jpg}
  \vspace{-4.5cm}
  \phrase{\hspace{10cm}\red{questions?}}
\end{frame}


%%%%%%%%%%%%%%%%%%%%%%%%%%%%%%%%%%%%%%%%%%%%%%%%%%%%%%%%%%%%%%%%%%%%%%
%\talksection{Break}

\talksection{Clustering}

\begin{frame}
  \frametitle{The Problem}
  Have points $d = \{d_1, \dotsc, d_n\}$.

  Have number of clusters $k$.

  \vspace{5mm}
  \textbf{Want:} an assignment of points to clusters
\end{frame}

\begin{frame}
  \cimggg{cluster-1.png}
\end{frame}

\begin{frame}
  \cimgh{cluster-2.png}
\end{frame}

\begin{frame}
  \cimg{cluster-3.png}
\end{frame}

\begin{frame}
  \cimg{cluster-4.png}
\end{frame}

\begin{frame}
  \frametitle{The Algorithm}
  \begin{enumerate}
  \item Assign points to clusters at random
  \item Repeat until stable:
  \begin{enumerate}
  \item Compute centroids of each cluster
  \item Assign points to nearest centroid
  \end{enumerate}
  \end{enumerate}
\end{frame}


\begin{frame}
  \frametitle{Cost function}
  \begin{mphrase}
    \textrm{cost} = \sum_i \sum_j \left| x_j - \mu_i \right|
  \end{mphrase}
\end{frame}

\begin{frame}
  \frametitle{Silhouette coefficient}
  \only<1-2>{
    Points $d = \{d_1, \dotsc, d_n\}$

    Clusters $K = \{c_1, \dotsc, c_k\}$.

    Cluster $c_{d_i}$ is the centroid of $d_i$.
  }
  
  \only<2>{
    \blue{Let $a_i$ be the average dissimilarity of $d_i$ to all
      points in its cluster.}

    \blue{Let $b_i$ be the least average dissimilarity of $d_i$ to any
      cluster other than $k_{d_i}$}
  }

  \only<3-5>{
    \begin{mphrase}
      s_i = \frac{b_i - a_i}{\max\{a_i, b_i\}}
    \end{mphrase}
  }
  
  \only<4-5>{
    \begin{mphrase}
      s_i = \left\{
        \begin{array}{ll}
          1 - a_i/b_i & \mbox{ if } a_i < b_i \\[2mm]
          0 & \mbox{ if } a_i = b_i \\[2mm]
          b_i / a_i - 1 & \mbox{ if } a_i > b_i
        \end{array}\right.
    \end{mphrase}
  }
  
  \only<5>{
    So $s_i \in [-1, 1]$
  }

  \only<6>{
    $s_i$ near 1 $\iff$ $d_i$ well clustered

    $s_i$ near 0 $\iff$ $d_i$ on the border between two clusters

    $s_i$ near -1 $\iff$ $d_i$ poorly clustered
  }

  \only<7>{
    \begin{mphrase}
      \textrm{Consider } \overline{s_i} \,\textrm{ over } i\in c_j \textrm{ for cluster } c_j
    \end{mphrase}
  }

  \only<8>{
    \begin{mphrase}
      \textrm{Consider } \overline{s_i}
    \end{mphrase}
  }
\end{frame}

\begin{frame}
  \vphrase{video time}
\end{frame}

\talksection{Anomaly Detection}

\begin{frame}
  \frametitle{Introduction to Anomaly Detection}
  \only<1>{
    \begin{itemize}
    \item Supervised
    \item Unsupervised
    \end{itemize}
  }
  \only<2>{
    Supervised anomaly detection:
    \begin{itemize}
    \item Training data: normal, abnormal
    \item Train a classifier
    \end{itemize}
    
    So reduced to existing problem of supervised classification.
  }
  \only<3>{
    Unsupervised anomaly detection:
    \begin{itemize}
    \item Mostly, this is clustering
    \item Increasingly, this is neural networks in advanced applications
    \end{itemize}
  }
  \only<4>{
    Applications:
    \begin{itemize}
    \item Intrusion detection (physical or electronic)
    \item Fraud detection
    \item Health monitoring (people, animals, machines)
    \end{itemize}
  }
  \only<5>{
    Techniques:
    \begin{itemize}
    \item Density: kNN, local outlier factor
    \item SVM
    \item Clustering: $k$-Means
    \end{itemize}
  }
  \only<6>{
    kNN techniques and variations
    \begin{itemize}
    \item Voronoi diagrams
    \item aNN
    \end{itemize}
  }
  \only<7>{
    LOF
    \begin{itemize}
    \item Measure average density using kNN
    \item Points with low local density are suspect outliers
    \item There is no good thresholding technique
    \end{itemize}
  }
  \only<8>{
    $k$-Means
  }
\end{frame}

\begin{frame}
  \frametitle{Examples}
  \only<1>{
    \vphrase{ping times}
  }
  \only<2>{
    \vphrase{httpd response times}
  }
  \only<3>{
    \vphrase{single/multiple host access abuse (DOS/DDOS)}
  }
  \only<4>{
    \vphrase{bank card fraud}
  }
  \only<5>{
    \vphrase{spam}
  }
\end{frame}


%%%%%%%%%%%%%%%%%%%%%%%%%%%%%%%%%%%%%%%%%%%%%%%%%%%%%%%%%%%%%%%%%%%%%%
%\talksection{Break}

\begin{frame}
  % https://www.pexels.com/photo/macro-shot-of-purple-flower-144282/
  % https://static.pexels.com/photos/144282/pexels-photo-144282.jpeg
  % CC0 license
  \cimgwb{blue-flowers.jpg}
  \vspace{-9cm}
  \wphrase{questions?}
\end{frame}


%%%%%%%%%%%%%%%%%%%%%%%%%%%%%%%%%%%%%%%%%%%%%%%%%%%%%%%%%%%%%%%%%%%%%%
%\talksection{Break}


\end{document}
